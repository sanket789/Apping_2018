\documentclass[8pt]{article}%
\renewcommand{\baselinestretch}{1} 
\usepackage{amsmath}%
\usepackage{amsfonts}%
\usepackage{amssymb}%
\usepackage{graphicx}
\usepackage{lipsum}
\usepackage{ragged2e}
\usepackage{bm}
\usepackage[none]{hyphenat}
\usepackage[margin = 20 mm]{geometry}
 \geometry{
 a4paper,
 total={210mm,297mm},
 left=20mm,
 right=20mm,
 top=20mm,
 bottom=10mm,
 }
\usepackage{enumerate}
\usepackage{hyperref}


%-------------------------------------------

\pagenumbering{gobble}
\begin{document}
\begin{center}
\huge{\textsc{Sanket Chirame}}\\
\vspace{1.5mm}
\normalsize{Department of Physics, IIT Bombay}\\
\end{center}
\vspace{3mm}
{\small
\begin{flushleft}
\bf{\Large{Scholastic Achievements}}
\end{flushleft}
\vspace{-1mm}
\hrule
\begin{itemize}
\setlength\itemsep{0.01em}%order check 

\item 
\item 
\item 

\end{itemize}

\begin{flushleft}
\bf{\Large{Key Projects}}
\end{flushleft}
 \vspace{-2mm}
\hrule

{\flushleft \bf \large{Star Formation Rate using Diffused Supernova Neutrino Background}}  \hfill {{{May '18 - Present}}} \\
{\em Guide: Prof. Vikram Rentala (Department of Physics, IIT Bombay)} 

\begin{itemize}
    \setlength\itemsep{0.01em}

    \item Simulated {\bf neutrino flux} from diffused supernova neutrino background for different detected energy bins in a {\bf neutrino detector} and validated the results from plots published by Shunsaku, et al.
    \item Critically examined the existing star formation theories with the observations of Super-K using chi-square hypothesis
    \item Reviewed the existing methods for measuring star formation rate and analyzed the {\bf best-fit models}

\end{itemize}

{\flushleft {\bf \large{Advitiy $|$  Student Satellite Project $|$ IIT Bombay}}} \hfill {{Feb '17 - Present}} 
\vspace{-2mm}
{\flushleft \em{The {\bf \em 2$^{nd}$ student satellite} of IIT Bombay, technically advanced and efficient version of the 1st, Pratham}}\\
\vspace{-6mm}

{\flushleft \bf {Attitude Determination and Control Subsystem}} 
   
   \begin{itemize}
   \setlength\itemsep{0.01em}
	\vspace{-2mm}
    \item Determined the {\bf attitude deviations} in an uncontrolled satellite using simulations of attitude dynamics in python
    \item Evaluated the use of {\bf photovoltaic modules} (used for solar panels) for sun vector measurement and concluded that the increased complexity of required electric circuit leads to rise in the failure probability of a satellite
    \item Analyzed the feasibility of different {\bf sun-vector measurement} technologies for our satellite
 \end{itemize}
    {\flushleft \bf {Payload Subsystem}} 
   
   \begin{itemize}
   \setlength\itemsep{0.01em}
	\vspace{-2mm}
	\item Explored different applications of {\bf cubesats based telescope} such as aerosol measurement, studying chromosphere, relationship between solar flares and energetic particles etc. and inspected their implementation in a satellite
    \item Analyzed {\bf Star-Tracker} to determine the feasibility of its application as an attitude sensor in \nohyphens{nanosatellites}
    \item Analyzed the electrical, mechanical and control law requirements for a cubesat to be used as a {\bf warning system for cyclones} following the guidelines of {\bf ISRO}

    \end{itemize}

{\flushleft \bf \large{Big Bang Nucleosynthesis}} \hfill {{{Jan '18 - Apr '18} }} \\
{\em Prof. Vikram Rentala (Department of Physics, IIT Bombay)} \hfill{\em Course Project - Astrophysics}
\begin{itemize}
\setlength\itemsep{0.01em}
 \item Theoretically predicted the abundances of {\bf H, D, He} and {\bf Li} in the universe using currently accepted {\bf Maxwell Boltzmann distribution} and then compared them to {\bf Tsallis nonextensive statistics} model
 \item Obtained an overview of {\bf big bang nucleosynthesis}, searched {\bf open problems} and further studied the problem of {\bf discrepancy} between predicted and measured amount of {\bf lithium}
\end{itemize}

{\flushleft \bf \large{Summer of Science $|$ Maths and Physics club $|$ IIT Bombay}}\\
{\em An initiative to boost knowledge of pure sciences where one can study any topic under the guidance of a proficient mentor
}
{\flushleft \bf {Relativity}} \hfill{{{May '18 - July '18}}} \\
\begin{itemize}
    \setlength\itemsep{0.01em}
	\vspace{-5mm}
    \item Studied {\bf special theory of relativity} and its implementation in mechanics and electromagnetism
    \item Grasped {\bf tensor analysis} and related mathematics to lay the foundation of general theory of relativity 
\end{itemize}

{\flushleft \bf {Cosmology}} \hfill{{{May '17 - June '17}}} \\
\begin{itemize}
    \setlength\itemsep{0.01em}
    \vspace{-5mm}
    \item Investigated {\bf basic principles} of cosmology such as Friedmann equation, cosmological constant etc.
    \item Used these principles to understand higher concepts of cosmology such as dark matter, early universe (nucleosynthesis and decoupling), inflation, baryogenesis etc. using {\bf Newtonian approach}
\end{itemize}

{\flushleft \bf \large{Disassembly of Washing Machine}} \hfill {{{Sep '18 - Nov '18} }} \\
{\em Prof. Parag Bhargav (Department of MEMS
, IIT Bombay)} \hfill{\em Course Project - Engineering Metallurgy}
\begin{itemize}
\setlength\itemsep{0.01em}
 \item Coordinated with a team of {\bf 15} to dismantle a washing machine to the smallest parts without its destruction
 \item Analyzed the composition of {\bf drum} and generated a report on involved materials and manufacturing processes

\end{itemize}

\newpage
\begin{flushleft}
\bf{\Large{Internship}}
\end{flushleft}
\vspace{-2mm}
\hrule

{\flushleft \bf \large{Aerostat for Military Surveillance}}  \hfill {{{Dec '17}}} \\
{\em {Manastu Space Technologies Private Limited} } 
\begin{itemize}
    \item Designed an {\bf aerostat}, a lighter than air aircraft and developed a two dimensional {\bf gore-profile} for the 3D design
    \item Manufactured {\bf prototype} using Low Density Polyethylene and experimentally determined the increase in lift for {\bf kytoon}, a tethered aircraft which is combination of a heavier than air kite and lighter than air balloon
\end{itemize}

\begin{flushleft}
\bf{\Large{Positions of Responsibility}}
\end{flushleft}
\vspace{-2mm}

\hrule

{\flushleft \bf \large{Subsystem Leader $|$ Control Subsystem $|$ Advitiy}} \hfill {{Feb '18 - Present}}\\
\vspace{-4mm}
\begin{itemize}
    
    \item Supervised a team of {\bf 8} people to develop {\bf quality assured} simulation frame-work for attitude dynamics of satellite
    \item Contributed to {\bf Satellite 101 wiki}, a compilation of basic knowledge of satellite project which reached {\bf 5.8k} pageviews and {\bf 1.4k} users around the globe within a month of launch
    \item Executed three step recruitment process to select {\bf 8} students for the subsystem from {\bf 50+} applicants evaluating their technical ability, practical approach and team work
\end{itemize}


{\flushleft \bf \large{Teaching Assistant}} \hfill {{{May '18 - June '18} }} \\ 
{\em Prof. D M Dewaikar, Department of Civil Engineering, IIT Bombay}
\begin{itemize}
    \setlength\itemsep{0.01em}
    \item Entrusted with the responsibility of tutoring {\bf 35} students for the course `{\bf Engineering Mechanics}'
    \item Helped students get better {\bf insight} of the course, clarified their doubts and assisted them in solving {\bf numericals}

\end{itemize}
 {\flushleft \bf \large{Associate Secretary }}  \hfill {{{Apr '17 - Mar '18} }} \\
 {\em Department of Mechanical Engineering, IIT Bombay}
\begin{itemize}
    \setlength\itemsep{0.01em}
    \item Responsible for facilitating the interaction of {\bf 140+ freshmen} with department faculties and seniors
    \item Successfully organized informal events like convocation, kurta day, trip etc. catering to {\bf 500+ students}
\end{itemize}


\begin{flushleft}
\bf{\Large{Technical Skills}}
\end{flushleft}
\vspace{-1mm}

\hrule
\vspace{1mm}
\begin{tabular}{l l}

\textbf{Micro-controller Programming} & Atmel Studio, Arduino \vspace{4pt}\\
\textbf{Languages} & C++, HTML, Python, \LaTeX  \vspace{4pt}\\

\textbf{Simulation and CAD Softwares} & SOLIDWORKS, AutoCAD, MATLAB 
	
\end{tabular}
\vspace{-4pt}

\begin{flushleft}
\bf{\Large{Relevant Courses}}
\end{flushleft}
\vspace{-1mm}
\hrule
\vspace{2mm}
\begin{tabular}{l l}

\textbf{Physics} & Quantum Physics and Application, Astrophysics, Basics of Electricity and Magnetism  \vspace{4pt}\\
\textbf{Mathematics} & Introduction to Numerical Analysis, Linear Algebra, Differential Equations, Calculus  \vspace{4pt}\\
\textbf{Other Courses} & Data Analysis and Interpretation, Nuclear Reactor Theory*, Microprocessors and Automatic Control*
	
\end{tabular}

\vspace{3mm}
\hspace{-5mm}
{\em * to be completed in November '18}
\begin{flushleft}
\bf{\Large{Extra-Curricular}}
\end{flushleft}
\hrule

{\flushleft \bf {Social Service}}
\begin{itemize}
    \item Taught {\bf basic mathematics} to students belonging to classes {\bf 6-9} at {\bf Abhyasika}, an NGO which provides free and quality education to {\bf underprivileged children} of nearby slums for {\bf 4 months}{\hfill \em (2017)}
    
    \item Assisted in execution of {\bf `SHE'} an initiative by {\bf Techfest} to promote Sanitary and Health Education through a network
of distribution of {\bf 200000+} sanitary pads through a route covering {\bf 37} remote villages {\hfill \em (2017)}

   \end{itemize}


{\flushleft \bf {Public Outreach}}
\begin{itemize}
\item Organized and conducted a Ground Station Workshop under Advitiy attended by {\bf 50+ students} and presented the use and working of {\bf rotor} and {\bf rotor-interface} for precise alignment of antenna  {\hfill \em (2017)}

    \item Exhibited {\bf Pratham} in engineers' conclave held at sixth Inter-IIT Technical Meet hosted by IIT Madras {\hfill \em (2017)}
     \end{itemize}


{\flushleft \bf {Sports}}
\begin{itemize}
    \item Attended Football Girls Camp and {\bf won} institute girls' {\bf football tournament} participated by 30+ girls     {\hfill \em (2017-18)}
   \item Completed year long training in Football under National Sports Organization {\hfill \em (2016-17)}
   \end{itemize}
}

\end{document}
