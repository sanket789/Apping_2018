\documentclass[10pt]{article}%
\renewcommand{\baselinestretch}{1} 
\usepackage{amsmath}%
\usepackage{amsfonts}%
\usepackage{amssymb}%
\usepackage{graphicx}
\usepackage{lipsum}
\usepackage{ragged2e}
\usepackage{bm}
\usepackage[usenames, dvipsnames]{color}
\definecolor{NavyBlue}{rgb}{0.0,0.0,0.40}
%\definecolor{NavyBlue}{rgb}{0.0,0.0,0.0}

\usepackage[none]{hyphenat}
\usepackage[margin = 20 mm]{geometry}
 \geometry{
 a4paper,
 total={210mm,297mm},
 left=20mm,
 right=20mm,
 top=20mm,
 bottom=15mm,
 }
\usepackage{enumerate}
\usepackage{hyperref}
\hypersetup{
    colorlinks=True,
    urlcolor=blue    } 
\urlstyle{same}     
\newcommand{\xfilll}[2][1ex]{
\dimen0=#2\advance\dimen0 by #1
\leaders\hrule height \dimen0 depth -#1\hfill}
%-------------------------------------------

\pagenumbering{gobble}
\begin{document}
\begin{center}
\huge{\textsc{Sanket Chirame}}\\
\vspace{1.5mm}
%\normalsize{Department of Physics, IIT Bombay}\\
\end{center}
Department of Physics  \hfill Mail: \href{mailto:sanketchirame12@gmail.com}{sanketchirame12@gmail.com}\\
Indian Institute of Technology, Bombay \hfill Phone: +91-9623541294\\
Mumbai, India $400 076$ \hfill Web: \url{https://sites.google.com/view/sanketchirame}\\
\vspace{2mm}
%\hrule
%\vspace{2mm}
{%\small
\begin{flushleft}
\Large\textsc{\textcolor{NavyBlue}{Research Interests\xfilll[2.5pt]{1.0pt}}}
\end{flushleft}
\vspace{-2mm}
\textbf{Theoretical Condensed Matter Physics:} Non-equilibrium physics of quantum systems, Phase transitions, Topological phases of quantum matter
\begin{flushleft}
\Large\textsc{\textcolor{NavyBlue}{Education \xfilll[2.5pt]{1.0pt}}}
\end{flushleft}
\vspace{-1mm}
Indian Institute of Technology, Bombay \hfill \textit{August 2019 (Expected)}\\
{\em B.Tech. + M.Tech. in Engineering Physics with specialization in \textbf{Nanoscience}}
%\end{itemize}



\begin{flushleft}
\Large\textsc{\textcolor{NavyBlue}{Research Experience \xfilll[2.5pt]{1.0pt}}}
\end{flushleft}
 \vspace{-3mm}

{\flushleft \bf \large{Master's Thesis - Non-equilibrium Physics of Many Body States }}  \hfill {{{\textit{July '18 - Present}}}} \\
{\em Guide: Prof. Soumya Bera, Department of Physics, IIT Bombay} 

\begin{itemize}
    \setlength\itemsep{0.01em}
    %\item Understood \textbf{Matrix Product state formalism} for efficient simulation of quantum many body states
    %\item Simulated quenching in transverse field Ising model using existing \textbf{DMRG} frame-work and studied non-analytic behaviour of Loschmidt echo corresponding to dynamical quantum phase transition
    %\item Studied \textbf{DMRG algorithm} for quench simulation of transverse field Ising model to study dynamical quantum phase transitions characterized by non-analytic behaviour of Loschmidt echo
    %\item Investigating Dynamic Quantum Phase Transitions in Transverse Field Ising Model using existing DMRG framework
    %\item Understood time dependent variational principle using Matrix Product state formalism
    \item Simulated nearest neighbour non-interacting fermion model with \textbf{Aubry-Andr´e potential} driven by external time dependent electric field and calculated correlation matrices
    \item Analysing \textbf{entanglement entropy} and charge current to understand the effect of drive frequency on phase transition from localized to delocalized phase of the model
    %\item Analysed the non-analyticity of the loschmidt echo to determine the correlation between phase transition and singular values
    \item Studied \textbf{DMRG} algorithm using \textbf{Matrix Product State} formalism to simulate time evolution of many body state of one dimensional lattice models
\end{itemize}

{\flushleft \bf \large{Summer Internship - Universit\"{a}t Konstanz, Germany}}  \hfill \textit{Summer 2017} \\
{\em Guide: Prof. Dr. Wolfgang Belzig, Dr. Akashdeep Kamra }\\
%\vspace{-6mm}
%{\flushleft \bf \normalsize{Qubit in Squeezed Boson Cavity\\}}
%{\em Project:  }\\
\vspace{-5mm}
\begin{itemize}
    \setlength\itemsep{0.01em}
    \item Studied effective Hamiltonian of \textbf{a qubit in photon cavity} system in dispersive regime important for experimental measurement of qubit state dependent photon occupation number
    \item Calculated first and second order corrections to interaction Hamiltonian in the presence of squeezing in the photonic mode using \textbf{time dependent perturbation theory}
    \item Analysed terms obtained \textbf{without} making \textbf{Rotating Wave Approximation} in the presence of squeezing
\end{itemize}



{\flushleft \bf \large{Dynamics of Cellular Networks}}  \hfill \textit{May '16 - Sep '16} \\
{\em Guide: Prof. Mandar Inamdar, Department of Civil Engineering, IIT Bombay} 

\begin{itemize}
    \setlength\itemsep{0.01em}
    \item Learnt \textbf{Chaste} C++ package enabling efficient simulations of cell monolayer as a vertex based model
    \item Studied dynamics of epithelial monolayer due to mechanical coupling of actomyosin cable contraction tensile force and cell crawling motile force
    \item Simulated crescent shaped wound in cell population to study the dynamics of boundary cells in the presence of \textbf{curvature dependent motile force} on the boundary vertices
    % \item Studied closure dynamics of wound in epithelial monolayer due to mechanical coupling of actomyosin cable contraction tensile force and cell crawling motile force
    % \item Performed simulations for cellular monolayer as a \textbf{vertex based model} using Chaste C$++$ library
    % \item Analyzed effect of \textbf{curvature dependent motile force} on the wound closure time and trajectories of boundary cells for crescent shaped wound
\end{itemize}


% {\flushleft \bf \large{Growth of Bacterial Cell Wall}}  \hfill {{{Jan '16 - May '16}}} \\
% {\em Guide: Prof. Anairban Sain, Department of Physics, IIT Bombay} 

% \begin{itemize}
%     \setlength\itemsep{0.01em}
%     \item Studied mechanism of addition of peptidoglycan strands in existing network during cell growth and its effect on the shape of cell
%     \item Understood mathematical model of cell wall growth, remodelling and division dynamics in Gram-negative bacteria during binary cell division
% \end{itemize}



\begin{flushleft}
\Large\textsc{\textcolor{NavyBlue}{Key Course Projects and Seminars\xfilll[2.5pt]{1.0pt}}}
\end{flushleft}
\vspace{-3mm}
{\flushleft \bf \large{Decoherence in Quantum Dots}} [\href{https://www.google.com/url?q=https%3A%2F%2Fbighome.iitb.ac.in%2Findex.php%2Fs%2FsdU2MtnBRNsytt9&sa=D&sntz=1&usg=AFQjCNGgNmMIDkn5vaH5aouBLhHL53AlJA}{Poster}]  \hfill \textit{Autumn 2017} \\
{\em Guide: Prof. Kasturi Saha, Department of Electrical Engineering, IIT Bombay \hfill Spintronics}
\vspace{-1.5mm}
\begin{itemize}
\setlength\itemsep{0.01em}
   % \item   Studied the \textbf{decoherence} of an electron spin in quantum dot due to interaction with a nuclear spin bath
   \item    Examined interplay of \textbf{Zeeman and Hyperfine interaction} to obtain an effective Hamiltonian for central electron spin in a quantum dot nuclear spin bath
    \item   Performed simulations to determine decoherence in InAs quantum dot using \textbf{pseudospin solution} and analysed the effect of external magnetic field on the decoherence time scale 
\end{itemize}



{\flushleft \bf \large{Spin-Orbit Coupling in Graphene}} [\href{https://www.google.com/url?q=https%3A%2F%2Fbighome.iitb.ac.in%2Findex.php%2Fs%2F5QcIL1HEGfFxNHu&sa=D&sntz=1&usg=AFQjCNFVORRMSdaonJwWcaGows0Odmuwxw}{Presentation}]  \hfill \textit{Spring 2017}\\
{\em Guide: Prof. Anshuman Kumar, Department of Physics, IIT Bombay \hfill Physics of nanostructures}
\vspace{-1.5mm}
\begin{itemize}
\setlength\itemsep{0.01em}
    \item Analysed band structure of graphene considering \textbf{spin-orbit coupling} in the presence of  electric field
    \item Studied the implications of time reversal symmetry on the degeneracy at the Dirac point
\end{itemize}

{\flushleft \bf \large{Introduction to String Theory}}   [\href{https://www.google.com/url?q=https%3A%2F%2Fbighome.iitb.ac.in%2Findex.php%2Fs%2Fh2gYWeT8Bbk2Wmr&sa=D&sntz=1&usg=AFQjCNFzDbrGThkVx7cXZQOkarlmLKg6bg}{Report}]  \hfill \textit{Autumn 2016} \\
{\em Guide: Prof. Kumar Rao, Department of Physics, IIT Bombay \hfill Supervised Learning}
\vspace{-1.5mm}
\begin{itemize}
\setlength\itemsep{0.01em}
    \item Studied the motion of classical relativistic strings using the \textbf{Nambu-Goto string action} and conserved currents arising from translational and Lorentz symmetries
    \item Developed an understanding of Gauss’ law and gravitational constant in extra compactified dimensions
\end{itemize}
\begin{flushleft}
\Large\textsc{\textcolor{NavyBlue}{Technical Experience \xfilll[2.5pt]{1.0pt}}}
\end{flushleft}
{\flushleft  \large{\textbf{Advitiy} - \href{https://www.aero.iitb.ac.in/satlab/}{Student Satellite Project, IIT Bombay}}}  \hfill {{{Feb '17 - Present}}} \\
{\em The $\mathit{2^{nd}}$ student satellite of IIT Bombay, technically advanced and efficient version of the $\mathit{1^{st}}$, Pratham} 

\begin{itemize}
    \setlength\itemsep{0.01em}
    \item Developed a \textbf{quality assured simulation frame-work} for attitude dynamics of satellite in \textbf{python} and performed
extensive simulations to determine attitude deviations in an uncontrolled satellite
    \item Determined the \textbf{feasible specifications for magnetorquer} (actuator) considering constraints imposed by all subsystems along with ensuring the successful detumbling of 1U satellite 
    \item Evaluated `Measuring Hardness ratio of Blackhole X-ray spectrum' as a potential payload idea for Advitiy considering on-board computational capabilities and X-ray detector specifications
\end{itemize}
{\flushleft \bf \large{Subsystem Head, ADC Subsystem, Advitiy}}  \hfill \textit{Feb' 17 - July' 18} \\
\begin{itemize}
\vspace{-4mm}
\setlength\itemsep{0.01em}
    \item Headed an \textbf{interdisciplinary team of 10 members} to generate a Baseline Design of Attitude Determination and Control Subsystem (ADCS) for Advitiy%head
    \item Executed \textbf{three stage recruitment process} to test technical skills, practical approach and team work of candidates thereby selecting 8 candidates out of 30 applicants%recruitment
    \item Developed and implemented \textbf{quality assurance guidelines} to make the design process more reliable 
    \item Contributed to \textbf{Satellite 101 wiki}, a compilation of exhaustive knowledge of satellite project which reached 5.8k
    page views and \textbf{1.4k users} around the globe within a month%wiki

\end{itemize}
\begin{flushleft}
\Large\textsc{\textcolor{NavyBlue}{Teaching Experience \xfilll[2.5pt]{1.0pt}}}
\end{flushleft}
{\flushleft \bf \large{Teaching Assistant - Microcontroller Lab}}  \hfill\textit{Autumn 2018} \\
\begin{itemize}
\vspace{-4mm}
\setlength\itemsep{0.01em}
    \item Entrusted with tutoring \textbf{$\mathbf{40+}$ students} for electronics lab based on \textbf{Arduino} programming 
    \item Assisting in design of lab assignments, solving experimental and theory doubts, and evaluating papers
    
\end{itemize}
\begin{flushleft}
\Large\textsc{\textcolor{NavyBlue}{Scholastic Achievements  \xfilll[2.5pt]{1.0pt}}}
\end{flushleft}
\vspace{-4mm}
\begin{itemize}
 \setlength\itemsep{0.01em}

\item Secured \textbf{All India Rank - 38} in Physics Graduate Aptitude Test in Engineering (GATE) \hfill\textit{2018} 
\item Awarded \textbf{AP grade} for exceptional performance in Physics of Nanostructures and Nanodevices \hfill\textit{2018}
\item Recipient of prestigious National Talent Search \textbf{(NTS)} scholarship  \hfill\textit{2010}
\item Secured \textbf{State Rank - 1} in Maharashtra state board secondary school examination \hfill\textit{2012}

\end{itemize}
\begin{flushleft}
\Large\textsc{\textcolor{NavyBlue}{Relevant Coursework \xfilll[2.5pt]{1.0pt}}}
\end{flushleft}
  
\begin{tabular}{ l p{13cm}}
    
    \textbf{Physics} & Theoretical Condensed Matter Physics, Superconductivity and Low Temperature Physics, Physics of Nanostructures, Relativistic Quantum Mechanics, Advanced Statistical Physics, Physics of Quantum Devices, Advanced Simulation Techniques \vspace{2mm}    \\ 
    \textbf{Mathematics} & Group Theory methods in Physics, Complex Analysis, Calculus, Numerical Analysis, Differential Equations (I \& II) \\ 
\end{tabular}
\begin{flushleft}
\Large\textsc{\textcolor{NavyBlue}{Technical Skills \xfilll[2.5pt]{1.0pt}}}
\end{flushleft}
%\vspace{-1mm}
\begin{tabular}{l l}

\textbf{Programming Languages} & Python, C++,  \LaTeX  \vspace{4pt}\\
\textbf{Simulation Softwares} & MATLAB, Mathematica, Simulink, Origin
\end{tabular} 

\begin{flushleft}
\Large\textsc{\textcolor{NavyBlue}{Extra-curricular Activities \xfilll[2.5pt]{1.0pt}}}
\end{flushleft}
\begin{itemize}
\vspace{-4mm}
\setlength\itemsep{0.01em}
    \item Devised modules to work with peer learning based pedagogy and flipped classroom model of teaching as an intern at \href{https://www.avanti.in/about/}{\textbf{Avanti Learning Centres}}, an emerging education company \hfill\textit{Summer 2015}
    \item Tutored $6^{th}$ standard students from NGO Vidya and LCCWA in Maths and Science as a part of \textbf{National Service Scheme} educational outreach program \hfill\textit{'14 - '15} 
    \item Secured \textbf{A grade} in both \textbf{Elementary} and \textbf{Intermediate} Drawing grade examinations conducted by the Maharashtra State Government \hfill\textit{'08 , '10}
    \item Completed three levels of \textbf{ICMAS Abacus} Mathematics program \hfill\textit{'08 -'09}
    
\end{itemize}

}
\end{document}
